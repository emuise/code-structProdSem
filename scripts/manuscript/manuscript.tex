% Options for packages loaded elsewhere
\PassOptionsToPackage{unicode}{hyperref}
\PassOptionsToPackage{hyphens}{url}
\PassOptionsToPackage{dvipsnames,svgnames,x11names}{xcolor}
%
\documentclass[
  authoryear,
  review,
  3p,
  twocolumn]{elsarticle}

\usepackage{amsmath,amssymb}
\usepackage{lmodern}
\usepackage{iftex}
\ifPDFTeX
  \usepackage[T1]{fontenc}
  \usepackage[utf8]{inputenc}
  \usepackage{textcomp} % provide euro and other symbols
\else % if luatex or xetex
  \usepackage{unicode-math}
  \defaultfontfeatures{Scale=MatchLowercase}
  \defaultfontfeatures[\rmfamily]{Ligatures=TeX,Scale=1}
\fi
% Use upquote if available, for straight quotes in verbatim environments
\IfFileExists{upquote.sty}{\usepackage{upquote}}{}
\IfFileExists{microtype.sty}{% use microtype if available
  \usepackage[]{microtype}
  \UseMicrotypeSet[protrusion]{basicmath} % disable protrusion for tt fonts
}{}
\makeatletter
\@ifundefined{KOMAClassName}{% if non-KOMA class
  \IfFileExists{parskip.sty}{%
    \usepackage{parskip}
  }{% else
    \setlength{\parindent}{0pt}
    \setlength{\parskip}{6pt plus 2pt minus 1pt}}
}{% if KOMA class
  \KOMAoptions{parskip=half}}
\makeatother
\usepackage{xcolor}
\setlength{\emergencystretch}{3em} % prevent overfull lines
\setcounter{secnumdepth}{5}
% Make \paragraph and \subparagraph free-standing
\ifx\paragraph\undefined\else
  \let\oldparagraph\paragraph
  \renewcommand{\paragraph}[1]{\oldparagraph{#1}\mbox{}}
\fi
\ifx\subparagraph\undefined\else
  \let\oldsubparagraph\subparagraph
  \renewcommand{\subparagraph}[1]{\oldsubparagraph{#1}\mbox{}}
\fi


\providecommand{\tightlist}{%
  \setlength{\itemsep}{0pt}\setlength{\parskip}{0pt}}\usepackage{longtable,booktabs,array}
\usepackage{calc} % for calculating minipage widths
% Correct order of tables after \paragraph or \subparagraph
\usepackage{etoolbox}
\makeatletter
\patchcmd\longtable{\par}{\if@noskipsec\mbox{}\fi\par}{}{}
\makeatother
% Allow footnotes in longtable head/foot
\IfFileExists{footnotehyper.sty}{\usepackage{footnotehyper}}{\usepackage{footnote}}
\makesavenoteenv{longtable}
\usepackage{graphicx}
\makeatletter
\def\maxwidth{\ifdim\Gin@nat@width>\linewidth\linewidth\else\Gin@nat@width\fi}
\def\maxheight{\ifdim\Gin@nat@height>\textheight\textheight\else\Gin@nat@height\fi}
\makeatother
% Scale images if necessary, so that they will not overflow the page
% margins by default, and it is still possible to overwrite the defaults
% using explicit options in \includegraphics[width, height, ...]{}
\setkeys{Gin}{width=\maxwidth,height=\maxheight,keepaspectratio}
% Set default figure placement to htbp
\makeatletter
\def\fps@figure{htbp}
\makeatother

\makeatletter
\makeatother
\makeatletter
\makeatother
\makeatletter
\@ifpackageloaded{caption}{}{\usepackage{caption}}
\AtBeginDocument{%
\ifdefined\contentsname
  \renewcommand*\contentsname{Table of contents}
\else
  \newcommand\contentsname{Table of contents}
\fi
\ifdefined\listfigurename
  \renewcommand*\listfigurename{List of Figures}
\else
  \newcommand\listfigurename{List of Figures}
\fi
\ifdefined\listtablename
  \renewcommand*\listtablename{List of Tables}
\else
  \newcommand\listtablename{List of Tables}
\fi
\ifdefined\figurename
  \renewcommand*\figurename{Figure}
\else
  \newcommand\figurename{Figure}
\fi
\ifdefined\tablename
  \renewcommand*\tablename{Table}
\else
  \newcommand\tablename{Table}
\fi
}
\@ifpackageloaded{float}{}{\usepackage{float}}
\floatstyle{ruled}
\@ifundefined{c@chapter}{\newfloat{codelisting}{h}{lop}}{\newfloat{codelisting}{h}{lop}[chapter]}
\floatname{codelisting}{Listing}
\newcommand*\listoflistings{\listof{codelisting}{List of Listings}}
\makeatother
\makeatletter
\@ifpackageloaded{caption}{}{\usepackage{caption}}
\@ifpackageloaded{subcaption}{}{\usepackage{subcaption}}
\makeatother
\makeatletter
\@ifpackageloaded{tcolorbox}{}{\usepackage[many]{tcolorbox}}
\makeatother
\makeatletter
\@ifundefined{shadecolor}{\definecolor{shadecolor}{rgb}{.97, .97, .97}}
\makeatother
\makeatletter
\makeatother
\usepackage{float}
\makeatletter
\let\oldlt\longtable
\let\endoldlt\endlongtable
\def\longtable{\@ifnextchar[\longtable@i \longtable@ii}
\def\longtable@i[#1]{\begin{figure}[H]
\onecolumn
\begin{minipage}{0.5\textwidth}
\oldlt[#1]
}
\def\longtable@ii{\begin{figure}[H]
\onecolumn
\begin{minipage}{0.5\textwidth}
\oldlt
}
\def\endlongtable{\endoldlt
\end{minipage}
\twocolumn
\end{figure}}
\makeatother
\journal{Remote Sensing of Environment}
\ifLuaTeX
  \usepackage{selnolig}  % disable illegal ligatures
\fi
\usepackage[]{natbib}
\bibliographystyle{elsarticle-harv}
\IfFileExists{bookmark.sty}{\usepackage{bookmark}}{\usepackage{hyperref}}
\IfFileExists{xurl.sty}{\usepackage{xurl}}{} % add URL line breaks if available
\urlstyle{same} % disable monospaced font for URLs
\hypersetup{
  pdftitle={Disentangling linkages between satellite derived forest structure and productivity essential biodiversity variables.},
  pdfauthor={Evan R. Muise; Margaret E. Andrew; Nicholas C. Coops; Txomin Hermosilla; A. Cole Burton; Stephen S. Ban},
  pdfkeywords={remote sensing, landsat, forest structure, forest
productivity, dynamic habitat indices, essential biodiversity
variables},
  colorlinks=true,
  linkcolor={blue},
  filecolor={Maroon},
  citecolor={Blue},
  urlcolor={Blue},
  pdfcreator={LaTeX via pandoc}}

\setlength{\parindent}{6pt}
\begin{document}

\begin{frontmatter}
\title{Disentangling linkages between satellite derived forest structure
and productivity essential biodiversity variables.}
\author[1]{Evan R. Muise%
\corref{cor1}%
}
 \ead{evan.muise@student.ubc.ca} 
\author[2]{Margaret E. Andrew%
%
}
 \ead{M.Andrew@murdoch.edu.au} 
\author[1]{Nicholas C. Coops%
%
}
 \ead{nicholas.coops@ubc.ca} 
\author[3]{Txomin Hermosilla%
%
}
 \ead{txomin.hermosillagomez@NRCan-RNCan.gc.ca} 
\author[1]{A. Cole Burton%
%
}
 \ead{cole.burton@ubc.ca} 
\author[4]{Stephen S. Ban%
%
}
 \ead{Stephen.Ban@gov.bc.ca} 

\affiliation[1]{organization={University of British Columbia, Forest
Resources Management},addressline={2424 Main Mall},city={Vancouver, BC,
Canada},postcode={V6T 1Z4},postcodesep={}}
\affiliation[2]{organization={Murdoch University, Environmental and
Conservation Sciences and Harry Butler Institute},city={Murdoch, WA,
Australia},postcode={6150},postcodesep={}}
\affiliation[3]{organization={Natural Resources Canada, Canada Forest
Service (Pacific Forestry Centre)},addressline={506 Burnside Rd
W},city={Victoria, BC, Canada},postcode={V8Z 1M5},postcodesep={}}
\affiliation[4]{organization={Ministry of Environment and Climate Change
Strategy, BC Parks},addressline={525 Superior Street},city={Victoria,
BC, Canada},postcode={V8V 1T7},postcodesep={}}

\cortext[cor1]{Corresponding author}






        
\begin{abstract}
{[}insert abstract here - nicholas ignore the formatting on the authors
for now, i can output to pdf and have it formatted nicely for
submission{]}
\end{abstract}





\begin{keyword}
    remote sensing \sep landsat \sep forest structure \sep forest
productivity \sep dynamic habitat indices \sep 
    essential biodiversity variables
\end{keyword}
\end{frontmatter}
    \ifdefined\Shaded\renewenvironment{Shaded}{\begin{tcolorbox}[boxrule=0pt, breakable, interior hidden, enhanced, borderline west={3pt}{0pt}{shadecolor}, frame hidden, sharp corners]}{\end{tcolorbox}}\fi

\hypertarget{introduction}{%
\section{Introduction}\label{introduction}}

With biodiversity being in decline, and facing extinction rates above
the background extinction rate \citep{thomas2004, urban2015}, as well as
the homogenization of communities at various scales \citep{mcgill2015}
it is integral to be able to monitor how biodiversity is changing across
the globe. In response, the global biodiversity community is making
efforts to assess and halt the degradation of biodiversity. The Group
for Earth Observation Biodiversity Observation Network has developed the
Essential Biodiversity Variables \citep[EBVs,][]{pereira2013}, designed
as an analog to the Essential Climate Variables framework
\citep{bojinski2014}. EBVs are designed to be global in scope, relevant
to biodiversity information, feasible to use, and complementary to one
another \citep{skidmore2021}. While it can be incredibly difficult, time
consuming, and expensive to collect data on biodiversity across wide
swaths of land and varying ecosystems, EBVs, which can be correlated to
sampled biodiversity information, allow for the monitoring and
asssessment of protected area effectiveness and ecosystem health at
large spatial scales \citep{hansen2021}. There are six EBV classes, each
of which correspond to a different facet of biodiversity including
species populations, species traits, community composition, ecosystem
structure, ecosystem function, and genetic composition
\citep{pereira2013}.

Satellite remote sensing has proven to be capable of measuring five of
the six EBV classes, the exception being genetic composition, which
requires in-situ observation and sampling \citep{skidmore2021}. Species
populations - and in turn community composition - can be assessed with
very-high-resolution imagery to identify tree species at the tree-crown
scale, however it is difficult and computationally expensive to extend
these analyses to broader extents \citep{fassnacht2016, graves2016}
while species traits such as vegetation phenology have been observed at
the single-tree scale using, for example, PlanetScope imagery and
drone-based measurements \citep{wu2021}. However, the spatially limited
and often \emph{ad-hoc} data collection approaches associated with
monitoring individuals is not conducive to the global or regional scales
required for biodiversity trend assessment \citep{valdez2023}.

The two landscape-level EBVs (ecosystem structure and function) are well
suited to be examined at large spatial scales using coarser spatial
measurements, such as those taken from satellites by the Moderate
Resolution Imaging Spectroradiometer {[}MODIS; \citet{zhang2003}{]}, the
Landsat imaging systems \citep{fisher2006}, or Sentinel-2
\citep{helfenstein2022, darvishzadeh2019} programs. These mid-resolution
satellites can monitor processes at broader extents but the coarse
spatial resolution removes the ability to relate these traits to
individual organisms. As a result, satellite remote sensing data has
been shown to be arguably the most effective at monitoring ecosystem
based EBVs focused on structure and function. These EBVs classes can be
monitored at regional to global extents through the use of optical
imagery \citep{cohen2004}, as well as active sensors such as lidar
(light detection and ranging) and radar
\citep{guo2017, lefsky2002, lang2021, neuenschwander2019, coops2016}.

\hypertarget{ebv---ecosystem-structure}{%
\subsection{EBV - Ecosystem Structure}\label{ebv---ecosystem-structure}}

Forest structural diversity has been linked to biodiversity at various
scales \citep{guo2017, bergen2009, gao2014}. Structural attributes range
in complexity from simple (canopy cover; canopy height), to more complex
(vertical and horizontal structural complexity) to modelled (aboveground
biomass; basal area), all of which can be assessed using lidar data
\citep{coops2021}. A suite of these lidar-derived attributes have been
used as local indicators of biodiversity, including simple metrics such
as canopy cover and canopy height as well as derived metrics including
vertical profiles, aboveground biomass
\citep{lefsky1999, guo2017, coops2016}. Other second order derived
metrics such as canopy texture, height class distribution, edges, and
patch metrics have also been used to examine habitat and biodiversity at
landscape scales \citep{bergen2009}. Advances in satellite remote
sensing processing have allowed 3D forest structure data to be imputed
across wide spatial scales \citep{matasci2018, coops2021} using data
fusion approaches involving collected lidar data and optical/radar data.

Increased forest structural complexity has been hypothesized to create
additional niches, leading to increased species diversity
\citep{bergen2009}, which has been frequently demonstrated using avian
species diversity metrics \citep{macarthur1961}. For example:
\citet{herniman2020} used spectral and lidar derived forest structure
data to model avian habitat suitability; \citet{clawges2008} found that
lidar derived forest structural attributes are capable of identifying
habitat types associated with avian species in pine/aspen forests;
\citet{goetz2007} used canopy structural diversity to predict bird
species richness, finding that canopy vertical distribution was the
strongest predictor of species richness. Forest structural metrics have
also been used to study biodiversity in other clades as well
\citep{davies2014, nelson2005}.

\hypertarget{ebv---ecosystem-function}{%
\subsection{EBV - Ecosystem Function}\label{ebv---ecosystem-function}}

With respect to ecosystem function, energy availability in an ecosystem
has shown to be a predictor of species richness and abundances at
various scales \citep{chase2002, radeloff2019, coops2019, razenkova},
and is measurable using satellite remote sensing via the use of various
vegetation indices \citep{huete2002, radeloff2019}. Vegetation indices,
which are indicative of photosynthetic activity, are commonly used as
proxies of gross primary productivity \citep{huang2019}. These
vegetation indices have also been used to assess patterns in
biodiversity at single time points \citep{bonn2004}, and more recently,
through yearly summaries of productivity
\citep{berry2007, radeloff2019}. The relationship between energy
availability and biodiversity occurs via various hypothesized
mechanisms, such as the available energy hypothesis
\citep{currie2004, wright1983}, the environmental stress hypothesis
\citep{currie2004}, and the environmental stability hypothesis
\citep{williams2008}. These three hypotheses have in turn been linked to
patterns of annual surface reflectance in remote sensing data
\citep{berry2007, radeloff2019}.

\citet{berry2007} first explored this idea by proposing the linkage of
intra-annual summaries of MODIS derived GPP to dispersive bird species.
This idea was further refined into the Dynamic Habitat Indices {[}DHIs;
\citet{coops2008}{]}, which have now been shown to be well suited to
assess the three aforementioned hypotheses at global scales
\citep{radeloff2019}. The cumulative DHI calculates the total amount of
energy available in a given pixel over the course of a year. Cumulative
DHI is strongly linked to the available energy hypothesis, which
suggests that with greater available energy species richness will
increase \citep{wright1983}. The minimum DHI, which calculates the
lowest productivity over the course of a year can be matched to the
environmental stress hypothesis, which proposes that higher levels of
minimum available energy will lead to higher species richness
\citep{currie2004}. Finally, the variation DHI, which calculates the
coefficient of variance in a vegetation index through the course of a
year, corresponds to the environmental stability hypothesis which states
that lower energy variation throughout a year will lead to increased
species richness \citep{williams2008}.

\hypertarget{biodiversity-monitoring-with-ebvs}{%
\subsection{Biodiversity Monitoring with
EBVs}\label{biodiversity-monitoring-with-ebvs}}

Biodiversity monitoring programs often require a range of information in
order to accurately assess changes in ecosystem integrity
\citep{lindenmayer2010}. Choosing datasets that are most closely related
to the phenomenon of interest in a given application allows for direct
connections to monitoring results and management actions
\citep{pressey2021}. With the advent of large-extent monitoring methods
like satellite remote sensing, and a proliferation of potential EBVs
datasets, it becomes important to assess the interrelationships between
these datasets, and assess their complementary of the information to
reduce duplication of efforts \citep{pereira2013, skidmore2021}. When
strong relationships are present between EBV classes, it becomes
possible to assess the ecological relationships between potential EBVs.
On the other hand, when datasets do not appear related, they may be well
suited to be used in monitoring programs together, as complementary
EBVs.

Linkages between forest ecosystem structure and function have been
examined within a remote sensing context for over 20 years
\citetext{\citealp[
\citet{knyazikhin1998}]{huete2002}; \citealp{myneni1994}}. While there
is significant theoretical and empirical evidence for their relationship
at single time points (within a single image) \citep{myneni1994},
various relationship directions and shapes have been found between
forest structure and function metrics \citep{ali2019}. Hypothesized
mechanisms such as niche complementary have shown that aboveground
biomass increases with stand structure \citep{zhang2012}, while
asymmetric competition for light can reduce forest productivity with
increased structural complexity \citep{bourdier2016}. The relationship
in particular between forest structural diversity metrics - which are
now more accurately and comprehensively derived from lidar data - and
temporal variation in functional metrics, specifically the metrics of
ecosystem productivity via the DHI framework, have yet to be fully
examined.

The overall goal of this paper is to assess patterns of forest ecosystem
structure and function and their complementarity across a wide range of
ecosystems encompassing significant environmental gradients. To do so,
we synthesize data from moderate-scale remote-sensing derived metrics of
forest structure, represented as both simple ALS-extracted metrics of
canopy height, cover and vertical complexity, as well as modelled forest
structure attributes including volume, and aboveground biomass, with a
well-established remote sensing derived index on ecosystem function. Our
first question is to examine how ecosystem structure and function
complement one another across a large environmental gradient and then
compare the simple and modelled representations of forest structure to
different levels of ecosystem function. This question is important as it
provides insights to the EBV community around complementarity of remote
sensing metrics when describing the structure and function of ecosystems
and proposes a method to examine potential overlap when generating
remote sensing EBVs.

Our second question examines the independent and shared relationships of
ecosystem structure height and cover, with modelled forest structure, on
ecosystem function. This provides insight into the choice of remote
sensing attributes to use when developing EBVs within a single EBV
class. Remote sensing datasets can comprise relatively unprocessed
observations, in this case ALS measures of height and cover which are
derived from the raw 3D point cloud vs modelled attributes, such as
biomass and volume, which involve the use the statistical relationships
with field data to transform the observations into more refined data
products. Assessing which of these two (or combination of the two)
approaches has stronger or weaker correlations with estimates of
function provides insights into the choice of data used to build EBVs.
Lastly, we examine how the primary and modelled structure attributes
partition the variance of the DHIs within key biomes and forest types
across a large environmental range, examining to what extent ecosystem
and forest types impacts these relationships and thus providing insight
into the applicability of these results globally.

\hypertarget{methods}{%
\section{Methods}\label{methods}}

\hypertarget{sec-study-area}{%
\subsection{Study Area}\label{sec-study-area}}

British Columbia is the westernmost province of Canada, and is home to a
variety of terrestrial ecosystems. Approximately 64\% of the province is
forested, with large environmental and topographic gradients
\citep{pojar1987, bcministryofforests2003}. The Biogeoclimatic Ecosystem
Classification (BEC) system identifies 16 zones based on the dominant
tree species and the ecosystems general climate. These zones can be
further split into subzones, variants, and phases based on microclimate,
precipitation, and topography \citep{pojar1987}. To examine trends
across the large environmental gradients, we group the BEC zones into
five broad biomes, specifically, the south interior, northern, montane,
alpine, and coastal groups similar to \citet{hamann2006}.

\{insert figure showing ecosystems of interest and grouped bec zones\}
need to remake figure showing ecosystems

\{table showing elevations and climate ranges for each included bec
zone\}

\hypertarget{data}{%
\subsection{Data}\label{data}}

\hypertarget{forest-structure}{%
\subsubsection{Forest Structure}\label{forest-structure}}

We used a suite of forest structural attributes (canopy height, canopy
cover, Lorey's height, overstory cover, basal area, aboveground biomass,
gross stem volume, mean elevation, elevation standard deviation, and
structural complexity {[}coefficient of variation in elevation
returns{]}). This dataset was created at a 30 m spatial resolution
according to \citet{matasci2018}. In brief, the method used a set of
lidar collections and field plots across Canada, and imputed the
remaining pixels using a random forest k-Nearest Neighbour approach on
Landsat-derived surface reflectance and auxiliary data such as
topography. Detailed information on the creation of this dataset can be
found in \citet{matasci2018}.

\hypertarget{dynamic-habitat-indices}{%
\subsubsection{Dynamic Habitat Indices}\label{dynamic-habitat-indices}}

We use an established set of indices of annual productivity shown to be
related to global biodvierstiy trends: the Dynamic Habitat Indices
{[}\citet{radeloff2019}{]}. The DHIs are a set of satellite remote
sensing derived productivity variables that summarize the cumulative
amount of available energy, the minimum available energy, and the
variation in available energy throughout a given year
{[}\citet{berry2007}; \citet{radeloff2019}{]}. The DHIs have previously
been produced at a global extent using MODIS imagery , and have been
used to assess alpha \citep{radeloff2019} and
beta{[}\citet{andrew2012}{]} diversity, species abundances
{[}\citet{razenkova}{]}, and construct novel ecoregionalizations
\citep{coops2009, andrew2013} . Recent studies have began to examine how
these indices can be constructed at a finer spatial resolution by using
multi-annual Landsat imagery to generate a single synthetic year of
monthly observations {[}\citet{razenkova2022}{]}.

The DHIs were calculated according to {[}\citet{razenkova2022};
RAZENKOVA LANDSAT DHI PAPER IN PRESS{]} for all of terrestrial British
Columbia. In brief, Google Earth Engine \citep{gorelick2017} was used to
obtain all valid Landsat pixels for a given study area, filtering out
pixels containing shadows, slouds, and cloud shadows within each image
\citep{zhu2012}, then calculated the NDVI for each pixel in each image.
They then calculated the median NDVI value for each month across the ten
year time span (2011-2020) to generate a synthetic year of monthly data.
The sum, minimum, and coefficient of variation across this synthetic
year of NDVI values is then calculated. More detailed information can be
found in \{Razenkova In Press\}.

\hypertarget{sampling}{%
\subsection{Sampling}\label{sampling}}

We conducted model-based sampling across the sixteen forest-dominated
ecosystems found within British Columbia {[}BC figure reference, I have
not made this figure yet{]}. Samples were randomly selected within each
BEC zone, in undisturbed pixels. Each sampled pixel had to have a
forested land cover class (coniferous, deciduous, mixed-wood, or
wetland-treed), and be surrounded by the same land cover class. The land
cover mask was generated by \citet[need to add to
zotero]{hermosilla2022} using a best-available-pixel composite, and an
inverse-distance weighted random forest approach across Canada.
Additionally, each pixel had to have a coefficient of variation less
than 0.5 in surrounding pixels in the two simplest forest structural
attributes, canopy height and canopy cover. A maximum of 3000 samples
were sampled in each BEC zone with a 1 km minimum sampling distance to
reduce the effects of spatial autocorrelation. All variables were
natural-log transformed and standardized. Variables containing zeros
were natural-log plus one transformed. Sampling was conducted in R
\citep{R-base} version 4.2.2 using the \textbf{sgsR} package
\citep{R-sgsR}. Focal analyses for the land cover classes and
coefficient of variation were calculated in Python version 3.9.

\hypertarget{analysis}{%
\subsection{Analysis}\label{analysis}}

\hypertarget{redundancy-analysis-and-variation-partitioning}{%
\subsubsection{Redundancy Analysis and Variation
Partitioning}\label{redundancy-analysis-and-variation-partitioning}}

Redundancy analysis (RDA) and variation partitioning were used to relate
the primary and modelled forest structure attributes to ecosystem
function across a broad environment range. Redundancy analysis functions
similarly to a multiple linear regression, except it is capable of
predicting multiple response variables. It accomplishes this by first
running a multiple linear regression of each predictor variable on each
response variable, then running a principle component analysis on the
residuals from each multiple linear regression. This reduces the
dimensionality of the output, and allows the relationship strength to be
assessed by calculating the loadings of both predictor and response
variables on the RDA axes. Partial redundancy analysis function
similarly, except also considers co-variates \citep{legendre2012}.
Redundancy analysis has widely been used in community ecology where
environmental variables of interest are compared to species composition
\citep{blanchet2014, kleyer2012}. While RDA is common in the ecological
literature, this analysis represents one of the first times this
technique has been applied to assess the complimentary of proposed EBVs.
Following the RDA, we employ ANOVAs to determine which axes are
significant, and calculate the proportion of variance attributable to
each axis using the eigenvalues generated from the RDA. We calculate
axis loadings for both predictor and response variables by calculating
the correlation between the variables and the RDA axes. Axis loadings
represent the relationship between a given variable and the RDA axis. We
only consider and display significant axes. To visualize the RDA for
both predictor and response variables, we display the results as path
diagrams, with loadings from each predictor to the RDA axis to the
response variables. The variance explained by each axis is also
displayed in the RDA box Figure~\ref{fig-rda-var}. All RDA calculations
were done in R \citep{R-base} version 4.2.2 using the \textbf{vegan}
package \citep{R-vegan}.

Variation partitioning is an extension of partial RDA which can assess
the overlap between the explanatory power of two datasets by utilizing
multiple partial RDAs and exchanging which datasets are considered the
predictor, and which is considered the co-variate \citep{legendre2012}.
Variation partitioning is traditionally displayed using a Venn diagram,
in which the percentage of variance explained by each dataset is in a
circle, and the overlap between circles represents the overlap in
variance explained. All variation partition analyses were done in R
\citep{R-base} version 4.2.2 using the \textbf{vegan} package
\citep{R-vegan}.

RDA and variation partitioning analyses were conducted for all samples,
as well as individually across each BEC zone and forest type. The
results were aggregated to BEC zone groups see~\ref{sec-study-area} for
visualization.

All code associated with the processing and analysis is available at
https://github.com/emuise/code-structProdSem.

\hypertarget{results}{%
\section{Results}\label{results}}

\begin{figure}

{\centering \includegraphics[width=2.66in,height=\textheight]{../../outputs/all_data_figure.drawio.png}

}

\caption{\label{fig-rda-var}A) Axis loadings from redundancy analysis of
primary and modelled forest structural attributes on the dynamic habitat
indices. B) Results from variation partitioning of primary and modelled
forest structural attributes on the DHIs. Both visualized analyses are
across all collected samples. See supplementary information for results
from each BEC zone and forest type.}

\end{figure}

Figure~\ref{fig-rda-var} A shows the results from redundancy analysis of
forest structural attributes on the dynamic habitat indices across the
entire sampled dataset. While there are three RDA axes associated with
the full dataset, the third axis explains 0.05\% of the variance in the
DHIs, and as such we do not show it. The first axis is strongly
represents all the DHIs (loadings \textgreater{} 0.85 for all DHIs), and
has the strongest loadings from canopy cover, basal area, aboveground
biomass, and gross stem volume. The other input attributes (canopy
height, structural complexity, and Lorey's height) have smaller
loadings. The second axis primarily represents the seasonality (Minimum
and Variation DHI) of the DHIs, with a very small loading on the
Cumulative DHI, and is primarily driven by canopy cover and complexity
(Figure~\ref{fig-rda-var} A).

The results from the variation partitioning (Figure~\ref{fig-rda-var} B)
show that the majority of the variance explained by the input datasets
is shared across both primary and modelled attributes. 14\% of the
variation in the DHIs is explained by primary and modelled attributes,
with 8.2\% of this being from overlap between the datasets. The primary
and modelled attributes explain 3 and 2.8\% of the variation on their
own, respectively (Figure~\ref{fig-rda-var} B).

\begin{figure}

{\centering \includegraphics[width=13.06in,height=\textheight]{../../outputs/radar_boxplot.png}

}

\caption{\label{fig-radar}Radar plots of average loadings strength by
group. A and B show input and response loadings, respectively. C)
Boxplots of BEC zone, forest types, and all data loadings for predictor
and response variables.}

\end{figure}

Figure~\ref{fig-radar} A and B show the axis loadings for predictor and
response variables, respectively. Across the BEC zone groups, the
loadings are generally similar in the predictor variables.
Figure~\ref{fig-radar} C shows the individual BEC zone loadings. Canopy
cover is generally the strongest predictor loading, and structural
complexity is generally the weakest predictor loading. The exception is
in deciduous forest types (mixed-wood and broadleaf), where structural
complexity is the largest axis loading, alongside canopy cover
(Figure~\ref{fig-radar} C). The second axis has smaller significant
loadings.

For predictor variables, there are differences between the two axes. The
first axis generally has large loadings across all DHIs, with the
exception being alpine ecosystems, where the minimum DHI has lower
loadings. The secondary RDA axis is primarily driven by variation in the
minimum DHI, with medium loadings in the variation DHI, and small
loadings in the cumulative DHI (Figure~\ref{fig-radar} B \& C).

\begin{figure}

{\centering \includegraphics[width=13.06in,height=\textheight]{../../outputs/raincloud.png}

}

\caption{\label{fig-raincloud}Raincloud plot of proportion of DHI
variation explained by total, the overlap between primary and modelled
structural attributes, and the overlap between primary and modelled
attributes.}

\end{figure}

Figure~\ref{fig-raincloud} shows raincloud plots of the percentage of
variation explained by predictor attributes in total, unique to modelled
attributes, unique to primary attributes, and the overlap between
primary and modelled attributes. Running the analysis by forest types
generally results in higher amounts of variance explained, this is
especially prevalent in deciduous forest types (mixed-wood and
broadleaf). The overlap between modelled and primary attributes is
generally higher than each dataset on their own
(Figure~\ref{fig-raincloud}). When the overlap explains a low amount
variance, the information represented by the primary and modelled
attributes are generally similar (i.e.~in the all data analysis, 14\% of
the DHI's variance is explained by the strucutral data, with 8\% being
from the overlap, and 3\% being from the primary and modelled
attributes, respectively; Figure~\ref{fig-rda-var} A). Notably, the
variation explained by the primary and modelled attributes is commonly
presented through a single RDA axis, which generally corresponds to
overall productivity through the year (Figure~\ref{fig-radar}).

\begin{figure}

{\centering \includegraphics[width=11.81in,height=\textheight]{../../outputs/con_var.png}

}

\caption{\label{fig-con-var}Proportion of variance explained by
proportion of BEC zone's forest that are coniferous.}

\end{figure}

Figure~\ref{fig-con-var} shows the proportion of variance explained as a
proportion of a given BEC zone's forest that is coniferous. As the
proportion of coniferous forest increases, so too does the amount of
variance explained, while analyses ran on each forest type do not show a
similar pattern (Figure~\ref{fig-raincloud}).

\begin{figure}

{\centering \includegraphics[width=8in,height=\textheight]{../../outputs/rgbplots_manual_edit.png}

}

\caption{\label{fig-fcc}False colour maps of axis loadings for the first
RDA axis (top) and second RDA axis (bottom). A and D show axis loadings
for canopy height (r), canopy cover (g) and structural complexity (b). B
and E show axis loadings for basal area(r), total biomass (g) and gross
stem volume (b). C and F show axis loadings for the cumulative DHI (r),
variation DHI (g) and minimum DHI (b).}

\end{figure}

Figure~\ref{fig-fcc} shows false colour composites of primary (A \& F),
modelled (B \& E), and DHI (C \& F) loadings across the BEC zones of
British Columbia. In the first RDA axis there is spatial variation in
the primary attributes, with the centre of the province's axis being
primarily driven by canopy cover (green), the coastal western hemlock
zone (southwest coast) having a strong structural complexity loading,
and the boreal in the northwest having the strongest loadings in canopy
height (Figure~\ref{fig-fcc} A). The modelled attributes generally show
gray-scale colour, indicating that basal area, total biomass, and gross
stem volume do not explain additional variation in the DHIs
(Figure~\ref{fig-fcc} B). Figure~\ref{fig-fcc} C shows the loadings of
the DHI, which are generally light tones, indicating high loadings
across the province. Yellow shows a smaller minimum DHI loading, with
higher loadings in the cumulative and variation DHIs.

Only four zones have a secondary RDA axis (Figure~\ref{fig-radar} C).
These zones have been spatially represented in (Figure~\ref{fig-fcc}
D-F). Boreal white and black spruce, interior Douglas-fir, mountain
hemlock, and montane spruce have a secondary axis, with no spatial
pattern associated with them. Canopy cover has the strongest loading in
the boreal white and black spruce, while the secondary axis in the
remaining three zones have the highest loadings in canopy height
(Figure~\ref{fig-fcc} D). Again, the modeleed attributes are in
grayscale, indicating similar loadings across the three modelled
attributes (Figure~\ref{fig-fcc} E). The DHIs in the second axis are
primarily driven by the minimum DHI, with low loadings on the cumulative
DHI, and moderate loadings on the variation DHI ( Figure~\ref{fig-fcc}
F; Figure~\ref{fig-radar} C).

\hypertarget{discussion}{%
\section{Discussion}\label{discussion}}

It is integral to consider the complementarity of potential EBVs
\citep{pereira2013, skidmore2021} . Forest structure and productivity
have been shown to be linked in multiple studies spanning three decades
\citep{ali2019, myneni1994}, however, a linkage between intra-annual
production and forest structure has not been shown. In this study, we
use statistical analyses from community ecology - namely redundancy
analysis and variation partitioning - to assess the complementarity of
forest structure and yearly productivity summaries. We find that the
datasets do not strongly overlap, with forest structure explaining 14\%
of the variation in the dynamic habitat indices in samples taken across
the entirety of British Columbia (Figure~\ref{fig-rda-var} B). This
indicates that they are suitable to be used in tandem with one another
when used as ecosystem EBVs.

Across most of British Columbia's ecosystem, we identified a single RDA
axis associated with the DHIs, encompassing the overall productivity
(Figure~\ref{fig-rda-var} A). Within this first axis, the strongest
loadings were canopy cover and the modelled attributes. When a second
axis was significant, it consistently had strong loadings on the minimum
and variation DHIs indicating a complex relationship within the DHIs in
certain ecosystems. This secondary axis has smaller axis loadings
associated with the primary and modelled attributes, with the strongest
loadings being canopy cover and structural complexity across the entire
dataset (Figure~\ref{fig-rda-var} A).

We also sought to explore whether modelled attributes (basal area, gross
stem volume, aboveground biomass, and Lorey's height) add additional
explanatory information when predicting the DHIs, as compared to primary
forest structural attributes. We generally found that canopy cover had
the largest axis loadings in the first RDA axis
(Figure~\ref{fig-radar}), which corresponds with canopy cover and LAI
intercepting radiation from the sun influencing the amount of energy
available in the environment \citep{knyazikhin1998}. Modelled attributes
such as basal area, aboveground biomass, and gross stem volume shared
similar loading magnitudes across the range of studied ecosystems,
indicating they do not add additional value when compared to one another
(Figure~\ref{fig-radar} C; Figure~\ref{fig-fcc} B). The loadings between
the modelled attributes and canopy cover are often similar, and as such,
we recommend utilizing the attributes derived directly from the point
cloud in the case of ALS data, or selecting a single modelled attribute.

During the analysis we noticed a high amount of total variance explained
in deciduous forests (mixed-wood and broadleaf) when compared to the
other two forest types and most BEC zones. This could possibly be due to
the loss of canopy cover during the winter being linked more closely to
the DHIs, which are an annual summary of productivity, and not taken at
a single time point, reducing the potential temporal mismatch between
the two datasets. Further, the strongest loadings in these two forest
types was vertical structural complexity, rather than canopy cover. This
could indicate that in deciduous forests, additional leaf density at
multiple layers is more important for productivity than canopy cover.

BEC zones dominated by coniferous forests tended to have higher amounts
of variance explained by the structural attributes
(Figure~\ref{fig-con-var}). This appears to be contrary to the
observation that mixed-wood and broadleaf forests have much higher
amounts of variation explained, and an analysis consisting solely of
coniferous pixels has relatively low amounts of variance explained
(Figure~\ref{fig-raincloud}). This could be due to the large range in
ecosystem variation across the province in coniferous forests. Running a
similar analysis across each forest type and each BEC zone could
increase the amount of explained variation, as the structural datasets
would likely have less internal variation.

Finally, we explored the amount of variation explained by primary vs
modelled attributes, as well as their overlap
(Figure~\ref{fig-raincloud}). We generally found that the overlap
between the primary and modelled attributes explained most of the
variation, with some exceptions, indicating that using either set of
forest structural attributes is suitable when monitoring biodiversity.

Recent advances in creating synthetic yearly observations have allowed
the DHIs to be generated at a Landsat scale (30 m), rather than the
previously used 250 m DHIs derived from MODIS {[}is the razenkova paper
out yet???; \citet{radeloff2019}; \citet{razenkova2022}{]}. This allows
these datasets to be matched and analyzed with other datasets generated
from the Landsat archive. This represents a significant advancement when
assessing the utility of EBVs, as the 30 m scale is well suited to
examine a range of ecological applications, including forest structure
and productivity \citep{cohen2004}.

In conclusion, we used redundancy analysis and variation partitioning to
assess the complimentarity of two potential EBV datasets - forest
structure and the DHIs. We also separated the forest structure datasets
into primary and modelled attributes in order to assess the need to
develop more complex structural attributes, or if base data derived
directly from lidar datasets was suitable. We found that the structural
attributes are not strongly related to the DHIs, indicating that they
are suitable to be used together as ecosystem scale EBVs when monitoring
forest environments. We also found that variation explained by the
overlap between primary and modelled attributes was often higher than
the variation explained by either individually.

Further research could assess the importance on intra-annual
productivity vs the single time point of forest strucutre, which does
not strongly change throughout the yaer (barring large disturbances).

\newpage


\renewcommand\refname{References}
  \bibliography{bibliography.bib,packages.bib}


\end{document}
