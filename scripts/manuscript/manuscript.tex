% Options for packages loaded elsewhere
\PassOptionsToPackage{unicode}{hyperref}
\PassOptionsToPackage{hyphens}{url}
\PassOptionsToPackage{dvipsnames,svgnames,x11names}{xcolor}
%
\documentclass[
  authoryear,
  review,
  3p,
  twocolumn]{elsarticle}

\usepackage{amsmath,amssymb}
\usepackage{lmodern}
\usepackage{iftex}
\ifPDFTeX
  \usepackage[T1]{fontenc}
  \usepackage[utf8]{inputenc}
  \usepackage{textcomp} % provide euro and other symbols
\else % if luatex or xetex
  \usepackage{unicode-math}
  \defaultfontfeatures{Scale=MatchLowercase}
  \defaultfontfeatures[\rmfamily]{Ligatures=TeX,Scale=1}
\fi
% Use upquote if available, for straight quotes in verbatim environments
\IfFileExists{upquote.sty}{\usepackage{upquote}}{}
\IfFileExists{microtype.sty}{% use microtype if available
  \usepackage[]{microtype}
  \UseMicrotypeSet[protrusion]{basicmath} % disable protrusion for tt fonts
}{}
\makeatletter
\@ifundefined{KOMAClassName}{% if non-KOMA class
  \IfFileExists{parskip.sty}{%
    \usepackage{parskip}
  }{% else
    \setlength{\parindent}{0pt}
    \setlength{\parskip}{6pt plus 2pt minus 1pt}}
}{% if KOMA class
  \KOMAoptions{parskip=half}}
\makeatother
\usepackage{xcolor}
\setlength{\emergencystretch}{3em} % prevent overfull lines
\setcounter{secnumdepth}{5}
% Make \paragraph and \subparagraph free-standing
\ifx\paragraph\undefined\else
  \let\oldparagraph\paragraph
  \renewcommand{\paragraph}[1]{\oldparagraph{#1}\mbox{}}
\fi
\ifx\subparagraph\undefined\else
  \let\oldsubparagraph\subparagraph
  \renewcommand{\subparagraph}[1]{\oldsubparagraph{#1}\mbox{}}
\fi


\providecommand{\tightlist}{%
  \setlength{\itemsep}{0pt}\setlength{\parskip}{0pt}}\usepackage{longtable,booktabs,array}
\usepackage{calc} % for calculating minipage widths
% Correct order of tables after \paragraph or \subparagraph
\usepackage{etoolbox}
\makeatletter
\patchcmd\longtable{\par}{\if@noskipsec\mbox{}\fi\par}{}{}
\makeatother
% Allow footnotes in longtable head/foot
\IfFileExists{footnotehyper.sty}{\usepackage{footnotehyper}}{\usepackage{footnote}}
\makesavenoteenv{longtable}
\usepackage{graphicx}
\makeatletter
\def\maxwidth{\ifdim\Gin@nat@width>\linewidth\linewidth\else\Gin@nat@width\fi}
\def\maxheight{\ifdim\Gin@nat@height>\textheight\textheight\else\Gin@nat@height\fi}
\makeatother
% Scale images if necessary, so that they will not overflow the page
% margins by default, and it is still possible to overwrite the defaults
% using explicit options in \includegraphics[width, height, ...]{}
\setkeys{Gin}{width=\maxwidth,height=\maxheight,keepaspectratio}
% Set default figure placement to htbp
\makeatletter
\def\fps@figure{htbp}
\makeatother

\makeatletter
\makeatother
\makeatletter
\makeatother
\makeatletter
\@ifpackageloaded{caption}{}{\usepackage{caption}}
\AtBeginDocument{%
\ifdefined\contentsname
  \renewcommand*\contentsname{Table of contents}
\else
  \newcommand\contentsname{Table of contents}
\fi
\ifdefined\listfigurename
  \renewcommand*\listfigurename{List of Figures}
\else
  \newcommand\listfigurename{List of Figures}
\fi
\ifdefined\listtablename
  \renewcommand*\listtablename{List of Tables}
\else
  \newcommand\listtablename{List of Tables}
\fi
\ifdefined\figurename
  \renewcommand*\figurename{Figure}
\else
  \newcommand\figurename{Figure}
\fi
\ifdefined\tablename
  \renewcommand*\tablename{Table}
\else
  \newcommand\tablename{Table}
\fi
}
\@ifpackageloaded{float}{}{\usepackage{float}}
\floatstyle{ruled}
\@ifundefined{c@chapter}{\newfloat{codelisting}{h}{lop}}{\newfloat{codelisting}{h}{lop}[chapter]}
\floatname{codelisting}{Listing}
\newcommand*\listoflistings{\listof{codelisting}{List of Listings}}
\makeatother
\makeatletter
\@ifpackageloaded{caption}{}{\usepackage{caption}}
\@ifpackageloaded{subcaption}{}{\usepackage{subcaption}}
\makeatother
\makeatletter
\@ifpackageloaded{tcolorbox}{}{\usepackage[many]{tcolorbox}}
\makeatother
\makeatletter
\@ifundefined{shadecolor}{\definecolor{shadecolor}{rgb}{.97, .97, .97}}
\makeatother
\makeatletter
\makeatother
\usepackage{float}
\makeatletter
\let\oldlt\longtable
\let\endoldlt\endlongtable
\def\longtable{\@ifnextchar[\longtable@i \longtable@ii}
\def\longtable@i[#1]{\begin{figure}[H]
\onecolumn
\begin{minipage}{0.5\textwidth}
\oldlt[#1]
}
\def\longtable@ii{\begin{figure}[H]
\onecolumn
\begin{minipage}{0.5\textwidth}
\oldlt
}
\def\endlongtable{\endoldlt
\end{minipage}
\twocolumn
\end{figure}}
\makeatother
\journal{Remote Sensing of Environment}
\ifLuaTeX
  \usepackage{selnolig}  % disable illegal ligatures
\fi
\usepackage[]{natbib}
\bibliographystyle{elsarticle-harv}
\IfFileExists{bookmark.sty}{\usepackage{bookmark}}{\usepackage{hyperref}}
\IfFileExists{xurl.sty}{\usepackage{xurl}}{} % add URL line breaks if available
\urlstyle{same} % disable monospaced font for URLs
\hypersetup{
  pdftitle={Disentangling linkages between satellite derived forest structure and productivity across a range of forest types and ecosystems.},
  pdfauthor={Evan R. Muise; Nicholas C. Coops; Txomin Hermosilla; A. Cole Burton; Margaret E. Andrew; Stephen S. Ban},
  pdfkeywords={structural equation modelling, remote
sensing, landsat, forest structure, forest productivity, dynamic habitat
indices},
  colorlinks=true,
  linkcolor={blue},
  filecolor={Maroon},
  citecolor={Blue},
  urlcolor={Blue},
  pdfcreator={LaTeX via pandoc}}

\setlength{\parindent}{6pt}
\begin{document}

\begin{frontmatter}
\title{Disentangling linkages between satellite derived forest structure
and productivity across a range of forest types and ecosystems.}
\author[1]{Evan R. Muise%
\corref{cor1}%
}
 \ead{evan.muise@student.ubc.ca} 
\author[1]{Nicholas C. Coops%
%
}
 \ead{nicholas.coops@ubc.ca} 
\author[2]{Txomin Hermosilla%
%
}
 \ead{txomin.hermosillagomez@NRCan-RNCan.gc.ca} 
\author[1]{A. Cole Burton%
%
}
 \ead{cole.burton@ubc.ca} 
\author[3]{Margaret E. Andrew%
%
}
 \ead{M.Andrew@murdoch.edu.au} 
\author[4]{Stephen S. Ban%
%
}
 \ead{Stephen.Ban@gov.bc.ca} 

\affiliation[1]{organization={University of British Columbia, Forest
Resources Management},addressline={2424 Main Mall},city={Vancouver, BC,
Canada},postcode={V6T 1Z4},postcodesep={}}
\affiliation[2]{organization={Natural Resources Canada, Canada Forest
Service (Pacific Forestry Centre)},addressline={506 Burnside Rd
W},city={Victoria, BC, Canada},postcode={V8Z 1M5},postcodesep={}}
\affiliation[3]{organization={Murdoch University, Environmental and
Conservation Sciences and Harry Butler Institute},city={Murdoch, WA,
Australia},postcode={6150},postcodesep={}}
\affiliation[4]{organization={Ministry of Environment and Climate Change
Strategy, BC Parks},addressline={525 Superior Street},city={Victoria,
BC, Canada},postcode={V8V 1T7},postcodesep={}}

\cortext[cor1]{Corresponding author}






        
\begin{abstract}
{[}insert abstract here - nicholas ignore the formatting on the authors
for now, i can output to pdf and have it formatted nicely for
submission{]}
\end{abstract}





\begin{keyword}
    structural equation modelling \sep remote
sensing \sep landsat \sep forest structure \sep forest
productivity \sep 
    dynamic habitat indices
\end{keyword}
\end{frontmatter}
    \ifdefined\Shaded\renewenvironment{Shaded}{\begin{tcolorbox}[interior hidden, boxrule=0pt, borderline west={3pt}{0pt}{shadecolor}, frame hidden, breakable, enhanced, sharp corners]}{\end{tcolorbox}}\fi

\hypertarget{introduction}{%
\section{Introduction}\label{introduction}}

Biodiversity is the summation of variation in biological life, across
genes, species, communities, and ecosystems. Currently, biodiversity is
in decline, facing extinction rates above the background extinction rate
\citep{thomas2004, urban2015}, and homogenization of communities at
various scales \citep{mcgill2015}. In response, the global biodiversity
community is making efforts to assess and halt the degradation of
biodiversity on earth. The Group for Earth Observation Biodiversity
Observation Network has developed the Essential Biodiversity Variables
\citep[EBVs,][]{pereira2013}, designed as an analog to the Essential
Climate Variables framework \citep{bojinski2014}. EBVs are designed to
be global in scope, relevant to biodiversity information, feasible to
utilize, and complementary to one another \citep{skidmore2021}.

There are six EBV classes, each corresponding to a different facet of
biodiversity, including, genetic composition, species populations,
species traits, community composition, ecosystem structure, and
ecosystem function \citep{pereira2013}. \citet{fernández2020} divides
the six classes into two approaches, with one focusing on species
biodiversity, and the other focusing on ecosystem diversity. Remote
sensing has proven to be capable of measure five of the six classes,
with the exception being genetic composition, which requires in-situ
observation and samples \citep{skidmore2021}. Notably, while remote
sensing can provide information on the remaining two species EBV
classes, this information is typically acquired with an \emph{ad-hoc}
approach requiring high spatial and spectral resolution data. This data
can often only be collected at local extents, rather than the global or
regional scales required for biodiversity trend assessment
\citep{valdez2023}. Community composition falls into a similar dilemma,
requiring species population information which necessitates high
resolution spatial data.

The remaining two classes, ecosystem structure and function, are
incredibly well suited to be examined at global or regional scales using
mid-resolution satellite imagery, such as that provided by the Landsat
series of satellites. Advances in satellite remote sensing processing
have allowed 3d forest structure data to be imputed across wide spatial
scales \citep{matasci2018, coops2021} using data fusion approaches
involving collected lidar data and optical/radar data. Other advances in
image compositing have allowed yearly summaries of vegetation
productivity to be calculated at regional to global scales, summarizing
the yearly energy totals, minimums, and variations \citep{radeloff2019}.
These datasets correspond quite well with the EBV classes ecosystem
structure (forest structural diversity metrics), and ecosystem function
(forest productivity metrics).

Forest structural diversity has been linked to biodiversity at various
scales \citep{guo2017, bergen2009, gao2014}. Increased structural
complexity is hypothesized to create additional niches, leading to
increased species diversity \citep{bergen2009}. The relationship between
forest structure and biodiversity is commonly assessed using avian
species diversity metrics \citep{macarthur1961, goetz2007}, however,
other clades (and habitats), have also been used
\citep{davies2014, nelson2005}. Many metrics derived from lidar remote
sensing have been used as local indicators of biodiversity, including
canopy cover, canopy height, vertical profiles, and aboveground biomass,
while other 2nd order derived metrics such as canopy texture, height
class distribution, edges, and patch metrics have been used to examine
habitat and biodiversity at landscape scales \citep{bergen2009}.

The dynamic habitat indices (DHIs) are indicators of productivity
calculated by summarizing vegetation indices over the course of one (or
multiple) years \citep{radeloff2019}. These indices have been related to
multiple facets of biodiversity at a range of scales, including species
occurrence and abundance \citep{razenkova2020}, alpha
\citep{radeloff2019} and beta diversity \citep{andrew2012}. Hypotheses
behind the biodiversity productivity relationships have been
established, including the species-energy hypothesis, the environmental
stress hypothesis, and the environmental stability hypothesis
\citep{coops2019}. The cumulative DHI calculates the total amount of
energy available in a given pixel over the course of a year. Cumulative
DHI is strongly linked to the available energy hypothesis, which
suggests that with greater available energy species richness will
increase \citep{wright1983}. The minimum DHI, which calculates the
lowest productivity over the course of a year can be matched to the
environmental stress hypothesis, which proposes that higher levels of
minimum available energy will lead to higher species richness
\citep{currie2004}. Finally, the variation DHI, which calculates the
coefficient of variance in a vegetation index through the course of a
year, corresponds to the environmental stability hypothesis which states
that lower energy variation throughout a year will lead to increased
species richness \citep{williams2008}.

Some vegetation structure metrics are simpler, and more accurate, to
calculate than others \citep{coops2021}. These basic metrics, such as
canopy height and canopy cover, can then be used to estimate additional
structure metrics, such as basal area or total biomass. Canopy height
and cover are commonly used as indicators of vertical and horizontal
variation, respectively. Recently, more attention has been paid to
internal structural complexity metrics, which can be more difficult and
time consuming to generate \citep{coops2021, ma2022}, but have been
shown to have stronger linkages with biodiversity \citep{guo2017}, and
productivity \citep{ali2019}.

Linkages between forest structure and productivity (namely, vegetation
indices) have been examined for nearly 20 years \citetext{\citealp[
\citet{knyazikhin1998}]{huete2002}; \citealp{myneni1994}}. While there
is significant theoretical and empirical evidence for their relationship
at single time points (within a single image) \citep{myneni1994},
various relationship directions and shapes have been found between
forest structure and productivity metrics \citep{ali2019}. These
relationships, their shapes, and their strengths have been attributed to
multiple possible hypotheses and have been shown to vary based on
environmental conditions \citep{ali2019}. The relationship between
forest structural diversity metrics and annual productivity summaries
has yet to be fully examined, including the DHIs.

Understanding the complementarity between potential EBVs is an integral
component of their creation \citep{skidmore2021}. Biodiversity has been
shown to be linked to both forest structure \citep{guo2017, gao2014},
and productivity \citep{radeloff2019} at a range of scales. Research has
highlighted that the relationship between productivity and biodiversity
is reciprocal \citep{worm2003}, making productivity a suitable
biodiversity indicator. In forested ecosystems, stand structural
attributes have also been shown to influence productivity, with a range
of responses dependent on environmental conditions \citep{ali2019}.

Structural equation modelling and path analyses have been commonly used
in ecology to assess the causal effects behind various hypotheses
\citep{fan2016, grace2010}, including the relationship between forest
structure, functioning, and biodiversity \citep{ali2019} . In this paper
we seek to untangle the relationship between two EBVs: forest structure
diversity metrics and yearly summaries of forest productivity. To
accomplish this, we assess this relationship using path analysis to
assess the direct and indirect (as mediated by more complex forest
structural diversity metrics) effects of commonly collected forest
structural metrics (canopy height and canopy cover) on yearly
productivity summaries. Further, we use exploratory structural equation
modelling (ESEM) to identify the latent variables driving the
productivity metrics. We run this analysis both globally, and stratified
by the forested ecosystems of British Columbia, Canada, and compare
models using AIC scores and global fit measures across the ecosystems.
The results from this study will assess the linkages and complementarity
between two biodiversity indicators.

\hypertarget{methods}{%
\section{Methods}\label{methods}}

\hypertarget{study-area}{%
\subsection{Study Area}\label{study-area}}

British Columbia is the westernmost province of Canada, and is home to a
variety of terrestrial ecosystems \citep{pojar1987}. Approximately 64\%
of the province is forested \citep{bcministryofforests2003}. There is a
large amount of ecosystem variation in the province, with large climate
and topographic gradients. The Biogeoclimatic Ecosystem Classification
(BEC) system identifies 16 zones based on the dominant tree species and
the ecosystems general climate. These zones can be further split into
subzones, variants, and phases based on microclimate, precipitation, and
topography \citep{pojar1987}.

\{insert figure showing ecosystems of interest\} need to remake figure
showing ecosystems, highlighting that bunchgrass isn't included

\{table showing elevations and climate ranges for each included bec
zone\}

\hypertarget{data}{%
\subsection{Data}\label{data}}

\hypertarget{ntems}{%
\subsubsection{NTEMS}\label{ntems}}

\hypertarget{dynamic-habitat-indices}{%
\subsubsection{Dynamic Habitat Indices}\label{dynamic-habitat-indices}}

The Dynamic Habitat Indices are a set of satellite remote sensing
derived productivity variables that summarize the cumulative amount of
available energy, the minimum available energy, and the variation in
available energy throughout a given year \citep{radeloff2019}. The DHIs
have previously been produced at a global extent using MODIS imagery
\citep{radeloff2019}, however, recent studies have examined how these
indices can be constructed at a finer spatial resolution using
multi-annual Landsat imagery.

\hypertarget{sampling}{%
\subsection{Sampling}\label{sampling}}

We conducted model based sampling across the fifteen forest dominated
ecosystems found within British Columbia {[}BC figure reference, I have
not made this figure yet{]}. Samples were randomly selected within BEC
zones of interest alongside multiple criteria. Each sampled pixel had to
have a forested land cover class (coniferous, deciduous, mixed-wood, or
wetland-treed), and be surrounded by the same land cover class.
Additionally, each pixel had to have a coefficient of variation less
than 0.5 in the Lorey's height and canopy cover forest structural
metrics. These focal analyses were preprocessed in R and Python, and
randomly sampled across a masked suitability raster for each BEC zone
using the \emph{sgsR} R package version 1.3.4 \citep{goodbody2023}. A
maximum of 3000 samples were taken from each BEC zone.

{[}note to ncc; I've stopped doing this sampling step as Meg said she
didn't feel it was necessary{]}

The imputation method used to generate the forest structure layers leads
to many duplicated values as there are a limited number of samples to
predict from \citep{matasci2018}. A discussion of imputation approaches
to large area lidar attribute modelling can be found in
\citet{coops2021}. To account for duplicated input values in our
modelling we averaged the endogenous variables within each ecosystem for
each uniquely imputed set of forest attributes, forest type, and
ecosystem. Due to variations in ecosystem size, differing numbers of
samples were generated for each ecosystem. To account for this during
model comparison, each ecosystem and land cover sample set was
downsampled to the minimum number of samples for a single BEC zone (in
this case, 170). For the forest type models, we attempted to select
equal numbers of observations from each BEC zone. However, some forest
types are rarely found in certain zones. To account for this, if an
equal number of samples from each BEC zone were not available, the
entire sample from that zone was taken. The amount of samples taken from
other zones was then increased in order to collect a total number equal
to the minimum number of samples in a single BEC zones (170). To meet
the normality assumptions of path analysis, all variables were
natural-log transformed and standardized, as per \citet{grace2016}.
Variables containing zeros were natural-log plus one transformed.

\hypertarget{analysis}{%
\subsection{Analysis}\label{analysis}}

\hypertarget{path-analysis}{%
\subsubsection{Path Analysis}\label{path-analysis}}

To determine the relationships between simple lidar derived forest
structural attributes, complex/derived forest structural attributes, and
forest productivity, we used path analysis to analyze two causal models
(Figure~\ref{fig-pathdag}). To determine primary drivers of the three
DHIs, we will summarize the predictors across ecosystems by counting the
strongest predictor in each BEC zone. This will determine if the primary
driver for each facet of the DHI is simple or complex, and allow us to
assess ecosystem differences.

\begin{figure}

{\centering \includegraphics[width=1.6in,height=\textheight]{../../outputs/path_dags.png}

}

\caption{\label{fig-pathdag}Proposed path diagrams}

\end{figure}

\hypertarget{exploratory-structural-equation-modelling}{%
\subsubsection{Exploratory Structural Equation
Modelling}\label{exploratory-structural-equation-modelling}}

Secondly, we will use exploratory structural equation modelling
\citep{marsh2020, asparouhov2009} to identify latent forest structural
variables within the data, and use these latent variables to predict the
dynamic habitat indices. Exploratory structural equation modelling is a
combination of exploratory factor analysis (EFA) and structural equation
modelling, which relaxes the strict requirement of zero cross-loadings
in confirmatory factor analysis and allows for less strict measurement
models to be used. ESEM is used when it is known that there is a latent
structure to the data, but the specific indicator variables have not yet
been determined.

One advantage of ESEM is that it can create varying numbers of latent
variables. For our analysis, we first ran exploratory factor analysis on
all forest structural attributes both globally and by BEC zone with
between 1-4 potential latent variables. We chose the most parsimonious
EFA model with the lowest AIC scores for each ecosystem, leading to
varying numbers of latent variables. We then determined the anchoring
indicator for each latent variable. The anchors were calculated by
choosing the indicator variable with the largest difference between the
maximum value in a given loading compared to to said indicators loadings
in all other latent variables. Each anchor variable was then assigned to
a named group (Canopy Cover, Height and Biomass, and Structural
Complexity).

Following the EFA, we used structural equation modelling, with latent
variables loadings determined by the EFA. We also filtered the loadings
to be greater than or equal to 0.5, allowing us to examine the number
and composition of latent variables found in each ecosystem. Following
this, we predicted the DHIs in a single SEM with covariances between the
DHIs. If a forest structural attribute did not end up in a latent
variable, it was included in the structural equation model as a DHI
predictor, without a latent variable as a mediator.

\hypertarget{results}{%
\section{Results}\label{results}}

\hypertarget{path-analysis-1}{%
\subsection{Path Analysis}\label{path-analysis-1}}

\begin{figure}

{\centering \includegraphics[width=14.74in,height=\textheight]{../../outputs/path_bars.png}

}

\caption{\label{fig-pathbar-prop}Proportion of strongest signficant
forest structural predictors of yearly productivity metrics in models
based on canopy cover vs canopy height.}

\end{figure}

\begin{figure}

{\centering \includegraphics[width=14.74in,height=\textheight]{../../outputs/strongest_predictor_bar.png}

}

\caption{\label{fig-pathbar-pred}Bar plot of strongest predictors in
path analysis for each BEC zone and predicted DHI variable.}

\end{figure}

\begin{figure}

{\centering \includegraphics[width=14.74in,height=\textheight]{../../outputs/path_map_edit.png}

}

\caption{\label{fig-pathmap}Map of strongest structural attribute
predictor strength on yearly productivity metrics by BEC zone for the
two proposed models. If there are no significant predictors, shown in
dark grey}

\end{figure}

\hypertarget{exploratory-structural-equation-modelling-1}{%
\subsection{Exploratory Structural Equation
Modelling}\label{exploratory-structural-equation-modelling-1}}

I also want to run exploratory SEM with and without the DHI's to see
what grouping it goes in. MinDHI has been breaking because some zones
only have 0 as mindhi (makes sense, they are in the alpine).

\begin{figure}

{\centering \includegraphics[width=14.74in,height=\textheight]{../../outputs/latent_groups.png}

}

\caption{\label{fig-esem-latent}Groups of latent variables with loadings
greater than or equal to 0.5 for each BEC zone. Loadings were determined
using exploratory factor analysis with up to four latent variables, and
selecting the number of possible latent variables with the lowest AIC.
All forest structure variables were potentially included as indicators
in each latent variable.}

\end{figure}

Exploratory factor analysis shows three groups of latent variables
(Figure~\ref{fig-esem-latent}). We assigned the groups to latent
variables indicating canopy cover / biomass metrics, canopy height /
biomass metrics, and structural complexity metrics. Indicators to latent
variable groups by examining anchors for each latent variable. The
anchors were calculated by choosing the indicator variable with the
largest difference between the maximum value in a given loading compared
to to said indicators loadings in all other latent variables.

Both the Ponderosa Pine (PP) and Sub-boreal Pine -- Spruce zones have
one latent variable. Coastal Western Hemlock (CWH), Engelmann Spurce --
Subalpine Fir (ESSF), and Montane Spruce (MS) are the only zones to have
structural complexity based latent variables. CWH is the only zone to
include elevation standard deviation in the structural complexity latent
variable.

\begin{figure}

{\centering \includegraphics[width=14.74in,height=\textheight]{../../outputs/latent_dhis.png}

}

\caption{\label{fig-esem-latent-dhi}Groups of latent variables with
loadings greater than or equal to 0.5 for each BEC zone, including the
DHIs. Loadings were determined using exploratory factor analysis with up
to five latent variables, and selecting the number of possible latent
variables with the lowest AIC. All forest structure variables were
potentially included as indicators in each latent variable.}

\end{figure}

Exploratory factor analysis shows three groups of latent variables. We
assigned these groups to attributes based on the indicator variables in
each latent variable, indicating canopy cover, height and biomass, and
structural complexity. When including the DHIs in this EFA, it is rare
that they are included with latent variables with large loadings. The
cumulative DHI is an indicator for the canopy cover latent varible in
the IDF (Interior Douglas-fir) and MS (Montane Spruce) BEC zones. The
DHIs are not included in any other latent variable, indicating that they
compose different information than the forest structural variables.

\begin{figure}

{\centering \includegraphics[width=14.74in,height=\textheight]{../../outputs/latent_boxplot.png}

}

\caption{\label{fig-latent-boxplots}Boxplots of latent variable loadings
across BEC zones as determined by exploratory factor analysis. Red
dashed lines show latent variable indicator cutoff before being used as
starting values for the SEM models.}

\end{figure}

Figure~\ref{fig-latent-boxplots} shows boxplots of latent variable
loadings from forest structure indicator variables. A maximum of three
valid latent variables were identified across the sixteen BEC zones. We
assigned indicators variables to latent variable groups by examining the
anchoring variables for each latent variable. The anchoring variables
were calculated by choosing the indicator variable with the largest
difference between the maximum value in a given loading compared to said
indicators loadings in all other latent variables. The only DHI variable
that was included in the latent variables after subsetting the latent
variables to only include strong loadings was the Cumulative DHI in the
canopy cover latent variable.

I need to actually make the DAGs and plot the parameters, but I haven't
gotten there yet. I'll plot the global one to show what it looks like,
then aggregate it in some way - probably similar to the path analysis. I
also may include the variable partitioning that Meg talked about in the
email if I can get it working.

\hypertarget{discussion}{%
\section{Discussion}\label{discussion}}

\hypertarget{conclusion}{%
\section{Conclusion}\label{conclusion}}

\newpage


\renewcommand\refname{References}
  \bibliography{bibliography.bib}


\end{document}
