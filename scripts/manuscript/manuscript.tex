% Options for packages loaded elsewhere
\PassOptionsToPackage{unicode}{hyperref}
\PassOptionsToPackage{hyphens}{url}
\PassOptionsToPackage{dvipsnames,svgnames,x11names}{xcolor}
%
\documentclass[
  authoryear,
  review,
  3p,
  twocolumn]{elsarticle}

\usepackage{amsmath,amssymb}
\usepackage{lmodern}
\usepackage{iftex}
\ifPDFTeX
  \usepackage[T1]{fontenc}
  \usepackage[utf8]{inputenc}
  \usepackage{textcomp} % provide euro and other symbols
\else % if luatex or xetex
  \usepackage{unicode-math}
  \defaultfontfeatures{Scale=MatchLowercase}
  \defaultfontfeatures[\rmfamily]{Ligatures=TeX,Scale=1}
\fi
% Use upquote if available, for straight quotes in verbatim environments
\IfFileExists{upquote.sty}{\usepackage{upquote}}{}
\IfFileExists{microtype.sty}{% use microtype if available
  \usepackage[]{microtype}
  \UseMicrotypeSet[protrusion]{basicmath} % disable protrusion for tt fonts
}{}
\makeatletter
\@ifundefined{KOMAClassName}{% if non-KOMA class
  \IfFileExists{parskip.sty}{%
    \usepackage{parskip}
  }{% else
    \setlength{\parindent}{0pt}
    \setlength{\parskip}{6pt plus 2pt minus 1pt}}
}{% if KOMA class
  \KOMAoptions{parskip=half}}
\makeatother
\usepackage{xcolor}
\setlength{\emergencystretch}{3em} % prevent overfull lines
\setcounter{secnumdepth}{5}
% Make \paragraph and \subparagraph free-standing
\ifx\paragraph\undefined\else
  \let\oldparagraph\paragraph
  \renewcommand{\paragraph}[1]{\oldparagraph{#1}\mbox{}}
\fi
\ifx\subparagraph\undefined\else
  \let\oldsubparagraph\subparagraph
  \renewcommand{\subparagraph}[1]{\oldsubparagraph{#1}\mbox{}}
\fi


\providecommand{\tightlist}{%
  \setlength{\itemsep}{0pt}\setlength{\parskip}{0pt}}\usepackage{longtable,booktabs,array}
\usepackage{calc} % for calculating minipage widths
% Correct order of tables after \paragraph or \subparagraph
\usepackage{etoolbox}
\makeatletter
\patchcmd\longtable{\par}{\if@noskipsec\mbox{}\fi\par}{}{}
\makeatother
% Allow footnotes in longtable head/foot
\IfFileExists{footnotehyper.sty}{\usepackage{footnotehyper}}{\usepackage{footnote}}
\makesavenoteenv{longtable}
\usepackage{graphicx}
\makeatletter
\def\maxwidth{\ifdim\Gin@nat@width>\linewidth\linewidth\else\Gin@nat@width\fi}
\def\maxheight{\ifdim\Gin@nat@height>\textheight\textheight\else\Gin@nat@height\fi}
\makeatother
% Scale images if necessary, so that they will not overflow the page
% margins by default, and it is still possible to overwrite the defaults
% using explicit options in \includegraphics[width, height, ...]{}
\setkeys{Gin}{width=\maxwidth,height=\maxheight,keepaspectratio}
% Set default figure placement to htbp
\makeatletter
\def\fps@figure{htbp}
\makeatother

\makeatletter
\makeatother
\makeatletter
\makeatother
\makeatletter
\@ifpackageloaded{caption}{}{\usepackage{caption}}
\AtBeginDocument{%
\ifdefined\contentsname
  \renewcommand*\contentsname{Table of contents}
\else
  \newcommand\contentsname{Table of contents}
\fi
\ifdefined\listfigurename
  \renewcommand*\listfigurename{List of Figures}
\else
  \newcommand\listfigurename{List of Figures}
\fi
\ifdefined\listtablename
  \renewcommand*\listtablename{List of Tables}
\else
  \newcommand\listtablename{List of Tables}
\fi
\ifdefined\figurename
  \renewcommand*\figurename{Figure}
\else
  \newcommand\figurename{Figure}
\fi
\ifdefined\tablename
  \renewcommand*\tablename{Table}
\else
  \newcommand\tablename{Table}
\fi
}
\@ifpackageloaded{float}{}{\usepackage{float}}
\floatstyle{ruled}
\@ifundefined{c@chapter}{\newfloat{codelisting}{h}{lop}}{\newfloat{codelisting}{h}{lop}[chapter]}
\floatname{codelisting}{Listing}
\newcommand*\listoflistings{\listof{codelisting}{List of Listings}}
\makeatother
\makeatletter
\@ifpackageloaded{caption}{}{\usepackage{caption}}
\@ifpackageloaded{subcaption}{}{\usepackage{subcaption}}
\makeatother
\makeatletter
\@ifpackageloaded{tcolorbox}{}{\usepackage[many]{tcolorbox}}
\makeatother
\makeatletter
\@ifundefined{shadecolor}{\definecolor{shadecolor}{rgb}{.97, .97, .97}}
\makeatother
\makeatletter
\makeatother
\usepackage{float}
\makeatletter
\let\oldlt\longtable
\let\endoldlt\endlongtable
\def\longtable{\@ifnextchar[\longtable@i \longtable@ii}
\def\longtable@i[#1]{\begin{figure}[H]
\onecolumn
\begin{minipage}{0.5\textwidth}
\oldlt[#1]
}
\def\longtable@ii{\begin{figure}[H]
\onecolumn
\begin{minipage}{0.5\textwidth}
\oldlt
}
\def\endlongtable{\endoldlt
\end{minipage}
\twocolumn
\end{figure}}
\makeatother
\journal{Remote Sensing of Environment}
\ifLuaTeX
  \usepackage{selnolig}  % disable illegal ligatures
\fi
\usepackage[]{natbib}
\bibliographystyle{elsarticle-harv}
\IfFileExists{bookmark.sty}{\usepackage{bookmark}}{\usepackage{hyperref}}
\IfFileExists{xurl.sty}{\usepackage{xurl}}{} % add URL line breaks if available
\urlstyle{same} % disable monospaced font for URLs
\hypersetup{
  pdftitle={Disentangling linkages between satellite derived forest structure and productivity across a range of forest types and ecosystems.},
  pdfauthor={Evan R. Muise; Nicholas C. Coops; Txomin Hermosilla; A. Cole Burton; Margaret E. Andrew; Stephen S. Ban},
  pdfkeywords={structural equation modelling, remote
sensing, landsat, forest structure, forest productivity},
  colorlinks=true,
  linkcolor={blue},
  filecolor={Maroon},
  citecolor={Blue},
  urlcolor={Blue},
  pdfcreator={LaTeX via pandoc}}

\setlength{\parindent}{6pt}
\begin{document}

\begin{frontmatter}
\title{Disentangling linkages between satellite derived forest structure
and productivity across a range of forest types and ecosystems.}
\author[1]{Evan R. Muise%
\corref{cor1}%
}
 \ead{evan.muise@student.ubc.ca} 
\author[1]{Nicholas C. Coops%
%
}
 \ead{nicholas.coops@ubc.ca} 
\author[2]{Txomin Hermosilla%
%
}
 \ead{txomin.hermosillagomez@NRCan-RNCan.gc.ca} 
\author[1]{A. Cole Burton%
%
}
 \ead{cole.burton@ubc.ca} 
\author[3]{Margaret E. Andrew%
%
}
 \ead{M.Andrew@murdoch.edu.au} 
\author[4]{Stephen S. Ban%
%
}
 \ead{Stephen.Ban@gov.bc.ca} 

\affiliation[1]{organization={University of British Columbia, Forest
Resources Management},addressline={2424 Main Mall},city={Vancouver, BC,
Canada},postcode={V6T 1Z4},postcodesep={}}
\affiliation[2]{organization={Natural Resources Canada, Canada Forest
Service (Pacific Forestry Centre)},addressline={506 Burnside Rd
W},city={Victoria, BC, Canada},postcode={V8Z 1M5},postcodesep={}}
\affiliation[3]{organization={Murdoch University, Environmental and
Conservation Sciences and Harry Butler Institute},city={Murdoch, WA,
Australia},postcode={6150},postcodesep={}}
\affiliation[4]{organization={Ministry of Environment and Climate Change
Strategy, BC Parks},addressline={525 Superior Street},city={Victoria,
BC, Canada},postcode={V8V 1T7},postcodesep={}}

\cortext[cor1]{Corresponding author}






        
\begin{abstract}
This is the abstract. Lorem ipsum dolor sit amet, consectetur adipiscing
elit. Vestibulum augue turpis, dictum non malesuada a, volutpat eget
velit. Nam placerat turpis purus, eu tristique ex tincidunt et. Mauris
sed augue eget turpis ultrices tincidunt. Sed et mi in leo porta
egestas. Aliquam non laoreet velit. Nunc quis ex vitae eros aliquet
auctor nec ac libero. Duis laoreet sapien eu mi luctus, in bibendum leo
molestie. Sed hendrerit diam diam, ac dapibus nisl volutpat vitae.
Aliquam bibendum varius libero, eu efficitur justo rutrum at. Sed at
tempus elit.
\end{abstract}





\begin{keyword}
    structural equation modelling \sep remote
sensing \sep landsat \sep forest structure \sep 
    forest productivity
\end{keyword}
\end{frontmatter}
    \ifdefined\Shaded\renewenvironment{Shaded}{\begin{tcolorbox}[breakable, sharp corners, interior hidden, frame hidden, enhanced, boxrule=0pt, borderline west={3pt}{0pt}{shadecolor}]}{\end{tcolorbox}}\fi

\hypertarget{introduction}{%
\section{Introduction}\label{introduction}}

Biodiversity is the summation of variation in biological life, across
genes, species, communities, and ecosystems. Currently, biodiversity is
in decline, facing extinction rates above the background extinction rate
\citep{thomas2004, urban2015}, and homogenization of communities at
various scales \citep{mcgill2015}. In response, the global biodiversity
community is making efforts to assess and halt the degradation of
biodiversity on earth. The Group for Earth Observation Biodiversity
Observation Network has developed the Essential Biodiversity Variables
\citep[EBVs,][]{pereira2013}, designed as an analog to the Essential
Climate Variables framework \citep{bojinski2014}. EBVs are designed to
be global in scope, relevant to biodiversity information, feasible to
utilize, and complementary to one another \citep{skidmore2021}.

There are six EBV classes, each corresponding to a different facet of
biodiversity, including, genetic composition, species populations,
species traits, community composition, ecosystem structure, and
ecosystem function \citep{pereira2013}. \citet{fernández2020} divides
the six classes into two approaches, with one focusing on species
biodiversity, and the other focusing on ecosystem diversity. Remote
sensing has proven to be capable of measure five of the six classes,
with the exception being genetic composition, which requires in-situ
observation and samples \citep{skidmore2021}. Notably, while remote
sensing can provide information on the remaining two species EBV
classes, this information is typically acquired with an \emph{ad-hoc}
approach. This collects high resolution, small spatial extent
information, rather than the global or regional scales required for
biodiversity assessment. Community composition falls into a similar
dilemma, requiring species population information which necessitates
high resolution spatial data.

The remaining two classes, ecosystem structure and function, are
incredibly well suited to be examined at global or regional scales using
mid-resolution satellite imagery, such as that provided by the Landsat
series of satellites. Advances in satellite remote sensing processing
have allowed 3d forest structure data to be imputed across wide spatial
scales \citep{matasci2018, coops2021} using data fusion approaches
involving collected lidar data and optical/radar data. Other advances in
image compositing have allowed yearly summaries of vegetation
productivity to be calculated at regional to global scales, summarizing
the yearly energy totals, minimums, and variations \citep{radeloff2019}.
These datasets correspond quite well with the EBV classes ecosystem
structure (forest structural diversity metrics), and ecosystem function
(forest productivity metrics).

Forest structural diversity has been linked to biodiversity at various
scales \citep{guo2017, bergen2009, gao2014}. Increased structural
complexity is hypothesized to create additional niches, leading to
increased species diversity \citep{bergen2009}. The relationship between
forest structure and biodiversity is commonly assessed using avian
species diversity metrics \citep{macarthur1961, goetz2007}, however,
other clades (and habitats), have been used
\citep{davies2014, nelson2005}. Many metrics derived from lidar remote
sensing have been used as local indicators of biodiversity, including
canopy cover, canopy height, vertical profiles, and aboveground biomass,
while other 2nd order derived metrics such as canopy texture, height
class distribution, edges, and patch metrics have been used to examine
habitat and biodiversity at landscape scales \citep{bergen2009}.

The dynamic habitat indices (DHIs) are indicators of productivity
calculated by summarizing vegetation indices over the course of one (or
multiple) years \citep{radeloff2019}. These indices have been related to
multiple facets of biodiversity at a range of scales, including species
occurrence and abundance \citep{razenkova2020}, alpha
\citep{radeloff2019} and beta diversity \citep{andrew2012}. Hypotheses
behind the biodiversity productivity relationships have been
established, including the species-energy hypothesis, the environmental
stress hypothesis, and the environmental stability hypothesis
\citep{coops2019}. The cumulative DHI calculates the total amount of
energy available in a given pixel over the course of a year. Cumulative
DHI is strongly linked to the available energy hypothesis, which
suggests that with greater available energy species richness will
increase \citep{wright1983}. The minimum DHI, which calculates the
lowest productivity over the course of a year can be matched to the
environmental stress hypothesis, which proposes that higher levels of
minimum available energy will lead to higher species richness
\citep{currie2004}. Finally, the variation DHI, which calculates the
coefficient of variance in a vegetation index through the course of a
year, corresponds to the environmental stability hypothesis which states
that lower energy variation throughout a year will lead to increased
species richness \citep{williams2008}.

Linkages between forest structure and productivity (namely, vegetation
indices) have been examined for nearly 20 years \citetext{\citealp[
\citet{knyazikhin1998}]{huete2002}; \citealp{myneni1994}}. While there
is significant theoretical and empirical evidence for their relationship
at single time points (within a single image) \citep{myneni1994},
various relationship directions and shapes have been found between
forest structure and productivity metrics \citep{ali2019}. These
relationships, their shapes, and their strengths have been attributed to
multiple possible hypotheses and can vary based on environmental
conditions \citep{ali2019}. The relationship between forest structural
diversity metrics and annual productivity summaries has yet to be fully
examined, including the DHIs.

Some vegetation structure metrics are simpler, and more accurate, to
calculate than others \citep{coops2021}. These basic metrics, such as
canopy height and canopy cover,can then be used to estimate additional
structure metrics, such as basal area or total biomass. Canopy height
and cover are commonly used as an indicator of vertical and horizontal
variation, respectively. Recently, more attention has been paid to
internal structural complexity metrics, which can be more difficult and
time consuming to generate \citep{coops2021, ma2022} .

In this study, we seek to untangle the relationship between two EBVs:
forest structure diversity metrics and yearly summaries of forest
productivity. To accomplish this, we assess this relationship using path
analysis to assess the direct and indirect (as mediated by more complex
forest structural diversity metrics) effects of commonly collected
forest structural metrics (canopy height and canopy cover) on yearly
productivity summaries. Doing so will in turn assess their utility as
complementary EBVs. We ran this analysis separately across four forested
land covers and the forested ecosystems of British Columbia, Canada.

\hypertarget{methods}{%
\section{Methods}\label{methods}}

\hypertarget{study-area}{%
\subsection{Study Area}\label{study-area}}

British Columbia is the westernmost province of Canada, and is home to a
variety of terrestrial ecosystems \citep{pojar1987}. Approximately 64\%
of the province is forested \citep{bcministryofforests2003}. There is a
large amount of ecosystem variation in the province, with large climate
and topographic gradients. The Biogeoclimatic Ecosystem Classification
(BEC) system identifies 16 zones based on the dominant tree species and
the ecosystems general climate. These zones can be further split into
subzones, variants, and phases based on microclimate, precipitation, and
topography.

\hypertarget{data}{%
\subsection{Data}\label{data}}

\hypertarget{sampling}{%
\subsection{Sampling}\label{sampling}}

\hypertarget{analysis}{%
\subsection{Analysis}\label{analysis}}

\hypertarget{results}{%
\section{Results}\label{results}}

\hypertarget{discussion}{%
\section{Discussion}\label{discussion}}

\hypertarget{conclusion}{%
\section{Conclusion}\label{conclusion}}

\newpage


\renewcommand\refname{References}
  \bibliography{bibliography.bib}


\end{document}
